\chapter{Conclusion}

The goal of this analysis was to (1) identify road characteristics that lead in an increased risk of accident and ultimately to (2) identify road segments that have more accidents than is expected given the traffic levels and road characteristics. Through fitting a Poisson regression, we were able to identify which road characteristics are most strongly associated with increased accident risk. By including a spatial component, we were able to account for both spatial correlations and unmeasured effects on road segments. These unmeasured effects allowed us to identify road segments with higher accidents than expected given the traffic levels and road characteristics. 

The spatial effects make it clear that the road characteristics alone do not contain all the information on number of accidents in a given road segment. There may be one or several road characteristics that haven't been included in the model that should be. One puzzling result seen above is that the effect of road characteristics differs drastically from road to road.  We hypothesize that this may be due to the fact that I-35E is a purely metropolitan interstate whereas I-35 spans urban and rural areas.  This suggests that, when considering systemic improvements, the type of improvement needed to prevent crashes depends on the type of interstate.

While this model presents a good start to identify road characteristics, the modeling can be improved.  Specifically, suggested next steps in risk factor analysis include, first, collect and use more covariates (road characteristics) in the model. From the results seen in Chapter 4, the spatial random effects are clearly capturing unobserved covariate data.  Particularly, including variables such as road curvature, grade (steepness), and whether or not an entrance or exit is present are important variables not included in this analysis (or the HSIS database) that affect crash risk. And, second, include cross-road spatial correlations. This analysis considered one road segment at a time.  However, I-35 and I-35E connect near the Twin Cities area.  Instead of making inference on one road at a time, including other roads which connect to or are close to the  road of interest may give more information on potential risk and preventative factors.
